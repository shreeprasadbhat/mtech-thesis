\section{Introduction}
The accelerating expansion of the universe is first found from the fact that the luminosity of type Ia supernovae (SNe Ia) is dimmer than expected \cite{riess1998observational}. This led to the discovery of Dark energy \cite{perlmutter1999measurements}. One of the few ways to measure properties of dark energy is to extend the Hubble Diagram(HD) to high redshift. The only way to extend HD to higher redshift is to Gamma Ray Burts (GRB). GRB have been found to be reasonably good standard candles in the usual sense that light curve and/or spectral properties are correlated to the luminosity, exactly as for Cepheids and supernovae, then simple measurements can be used to infer their luminosities and hence distances. The default expectation is the simplest model for the Dark Energy, where it does not change in time. This can be parametrized with the equation of state of the Dark Energy. The concordance case has w = -1 at all times, and this is the expectation of Einstein’s cosmological constant, or if the Dark Energy arises from vacuum energy. Given the strong results from supernovae for redshifts of less than 1, the frontier has now been pushed to asking the question of whether the value of w changes with time (and redshift).

The best way to measure properties of the Dark Energy seems to be to measure the expansion history of our Universe and place significant constraints on models of the Universe. Hubble diagram can be used to measure it. The Hubble diagram (HD) is a plot of distance versus redshift, with the slope giving the expansion history of our Universe. been proposed to determine the distances and redshifts of two thousand supernovae per year out to redshift 1.7 with exquisite accuracy. The default expectation is the simplest model for the Dark Energy, where it does not change in time. This can be parameterized with the equation of state of the Dark Energy. The best way to measure whether dark energy changed with respect to redshift, is to measure it over wide range of redshifts, but supernovaes cannot be deteceted above 1.7 even with modern satelites. But GRBs offer means extend HD over redshift > 6. The reason is that GRBs are visible across much larger distances than supernovae.

GRBs are now known to have several light curve and spectral properties from which the luminosity of the burst can be calculated (once calibrated), and these make GRBs into ’standard candles’. Several interesting correlations among Gamma Ray Burst (GRB) observables with available redshifts have been recently identified. Proper evaluation and calibration of these correlations may facilitate the use of GRBs as standard candles constraining the expansion history of the universe up to redshifts of $z > 6$.