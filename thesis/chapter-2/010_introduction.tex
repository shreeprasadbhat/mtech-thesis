\section{Introduction}
\cite{riess1998observational} and \cite{perlmutter1999measurements} discovered that luminosity of Type Ia Supernovae are fainter than expected for decelerating Universe. This lead to conclusion that universe expansion is accelerating. Dark energy is proposed to account for this accelerating expansion, and its makes $73$\% of universe. Other observations from Cosmic Microwave Background(CMB)\cite{aghanim2020planck} and Baryon Acoustic Oscillations(BAO)\cite{alam2017clustering} also supports this claim. The study of the nature of dark energy has become one of the most important issues in the field of fundamental physics. The simplest model for dark energy is $\Lambda$CDM, where $\Lambda$ is the cosmological constant, which is equivalent to the quantum vacuum energy. For $\Lambda$CDM, the equation of state parameter is $w = -1$, so $p = -\rho$. $\Lambda$CDM model is very popular and accepted, since it can explain the current various astronomical observations quite well. But the cosmological constant has always been facing the severe theoretical challenges, such as the fine-tuning and coincidence problems. Hence other possible models are proposed. For example, a spatially homogeneous, slowly rolling scalar field can also provide a negative pressure, driving the cosmic acceleration\cite{zlatev1999quintessence}. More generally, one can phenomenologically characterize the property of dynamical dark energy through parametrizing $w$ of its equation of state (EoS) $p = -w\rho$, where $w$ is usually called the EoS parameter of dark energy. For example, the simplest parametrization model corresponds to the case of $w=constant$, and this cosmological model is sometimes called the $\omega$CDM model. A more physical and realistic situation is that $w$ is time variable, which is often probed by the so-called Chevalliear–Polarski–Linder (CPL) parametrization, $w(a)=w_0+w_a(1-a)$ \cite{chevallier2001accelerating}. 