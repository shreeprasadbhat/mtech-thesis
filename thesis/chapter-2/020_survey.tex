\section{Literature Survey}
Given a specific model and a set Then a question of model choice naturally arises with the development of various dark energy models. A variety of methods such as the $F$-test, Akaike information criterion (AIC) (Penny et al. 2006), Mallows $C_{p}$, Bayesian information criterion (BIC) (Penny et al. 2006), minimum description length (MDL) (Rissanen 1978), and Bayesian model averaging have been proposed to select a good or useful model in light of observations. MacKay 1992 strongly recommends using Bayesian evidence to assign preferences to alternative models since the evidence is the Bayesian's transportable quantity between models, and the popular easyto-use AIC and BIC as well as MDL methods are all approximations to the Bayesian evidence (Penny et al. 2006). The Bayesian evidence for model selection has been applied to the study of cosmology for a long time (Trotta 2008; Martin et al. 2011; Lonappan et al. 2018; Basilakos et al. 2018), and recently a detailed study of Bayesian evidence for a large class of cosmological models taking into account around 21 different dark energy models has been performed by Lonappan et al. 2018. Although Bayesian evidence remains the preferred method compared with information criterions, a full Bayesian inference for model selection is very computationally expensive and often suffers from multi-modal posteriors and parameter degeneracies, which lead to a large time consumption to obtain the final result. \cite{li2019model} explored deep learning for model comparison.