\documentclass[12pt]{article}
\title{project-1}
\author{Shreeprasad Bhat}
\begin{document}
\maketitle
\section{Introduction}
The accelerating expansion of the universe is first found from the fact that the luminosity of type Ia supernovae(SNe Ia) is dimmer than expected (Riess et al. 1998; Permutter et al. 1999). Since then many SNe Ia datasets have been compiled to cosmological researchers (Suzuki et al 2012; Betoule et al. 2014; Scolinic et al 2018). However, due to the limited luminosity, most observable SNe Ia are at redshift z < 2. With these low redshift data, cosmological models cannot be unambiguously distinguished (Zhu et al. 2008; Dolgov et al. 2014; Tutusaus et al. 2016; Wei et al. 2016). Among these SNe Ia samples, the most up-to-date Pantheon compilation (Scolnic et al. 2018) is the largest sample and the redshift of the furthest SNe reaches up to z 2.3, but the number of SNe whose redshift is larger than 1.4 is only
six. Actually, the subtle difference between cosmological models in low redshift range would be remarkable in high redshift range. For various cosmological models, such as ΛCDM model, wCDM model, Chevallier-Polarsky-Linder (CPL) model, holographic dark energy (HDE) model and generalised chaplygin gas (GCG) model, the evolutions of dark energy equation of state are discrepant in high redshift (Chevallier & Polarski 2001; Wang et al. 2017; Pan et al.2020; Escamilla-Rivera et al. 2020). Thus, it is important to extend the Hubble diagram to high redshift range. Gamma-ray bursts (GRBs), as the most energetic explosions in the universe, is detectable up to very high redshift (Cucchiara et al. 2011). Therefore, it is possible to use GRBs as standard candles to trace the Hubble diagram at high redshift. Combining GRBs with other standard candles, the cosmological parameters can be tightly constrained (Friedman & Bloom 2005; Wang & Dai 2006; Izzo et al. 2009; Liang et al. 2011; Bloom et al. 2003; Xu et al. 2005; Wei & Zhang 2009; Demianski & Piedipalumbo 2011; Cai et al. 2013; Chang et al. 2014; Lin et al. 2016). However, it is not easy to calibrate the distance of GRBs due to the lack of knowledge on the explosion mechanism.
\end{document}
